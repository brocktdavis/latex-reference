\documentclass{article}

% Packages (these must be included in preamble)
\usepackage{amsmath}
\usepackage{graphicx}
\usepackage{subcaption}
\usepackage{setspace}

%\setcounter{tocdepth}{1} % Show sections
%\setcounter{tocdepth}{2} % + subsections
\setcounter{tocdepth}{3} % + subsubsections
%\setcounter{tocdepth}{4} % + paragraphs
%\setcounter{tocdepth}{5} % + subparagraphs

% Preamble
\title{Brock's LaTeX learning}
\date{2020-11-29}
\author{Brock T Davis}

\begin{document}

  % Title page
  \maketitle
  \pagenumbering{gobble}
  \newpage

  % Table of Contents
  \doublespacing
  \tableofcontents
  \singlespacing
  \newpage


  \pagenumbering{arabic}

  % ---------
  % Section 1
  % ---------
  \section{Doing some math stuff right here}
    Time to learn some LaTeX!

  \subsection{Some functions}

    A good ole function:
    \begin{equation}
      f(x) = x^2
    \end{equation}

    An unnumbered function
    \begin{equation*}
      f_2(x) = x^3
    \end{equation*}

  \subsubsection{Some more functions}
    An example of an inline equation would be something like $f(x) = x ^3 + 3x^2 + 33$ and it's really neat!

    \noindent
    Also, align environment is really cool to align stuff.
    For example:
    \begin{align*}
      1 + 3 &= 3 \\
      1 &= 3 - 2
    \end{align*}

    \noindent
    Or:
    \begin{align*}
      f(x) &= x^2 \\
      g(x) &= \frac{1}{x} \\
      F(x) &= \int^a_b \frac{1}{3}x^3
    \end{align*}

  \subsubsection{Matrix Time!}
    Matrix no brackets:
    \begin{align*}
    \begin{matrix}
      1 & 0 \\
      0 & 1
    \end{matrix}
    \end{align*}

    \noindent
    Matrix wrong brackets:
    \begin{equation*}
    [
    \begin{matrix}
      1 & 0 \\
      0 & 1
    \end{matrix}
    ]
    \end{equation*}

    \noindent
    Matrix correct brackets:
    $
    \left[
    \begin{matrix}
      1 & 0 \\
      0 & 1
    \end{matrix}
    \right]
    $

  \paragraph{Pp}
    Some more text.

  \paragraph{pp2}
    Some more more text.

  \subsection{Somethin else}
    Hey-O

  \paragraph{Yet another thing}
    Text time!

  % ---------
  % Section 2
  % ---------
  \section{Figure Stuff}
    Doing some images and figures here

  % Setting float can be done with \begin{figure}[...]
  % h (here) - same location
  % t (top) - top of page
  % b (bottom) - bottom of page
  % p (page) - on an extra page
  % ! (override) - will force specified location
  \begin{figure}[h!]
    \includegraphics[width=\linewidth]{boat.jpeg}
    \caption{A boat.}
    \label{fig:boat1}
  \end{figure}

  \begin{figure}[h!]
    \includegraphics[width=\linewidth]{boat.jpeg}
    \caption{The same boat}
    \label{fig:boat2}
  \end{figure}

  \begin{figure}[b]
    \centering
    % These linewidth should be 0.1 less than what you expect
    \begin{subfigure}[b]{0.4\linewidth}
      \includegraphics[width=\linewidth]{boat.jpeg}
      \caption{Boaty McBoatface.}
    \end{subfigure}
    \begin{subfigure}[b]{0.4\linewidth}
      \includegraphics[width=\linewidth]{boat.jpeg}
      \caption{Boaty McBoatface 2.}
    \end{subfigure}
    \caption{An armada of boats.}
    \label{fig:boats}
  \end{figure}

  Figure \ref{fig:boat1} shows a boat.

  % ---------
  % Section 3
  % ---------
  \section{Tables}
    \begin{table}
      \caption{Dummy Table}
    \end{table}


  % --------
  % Appendix
  % --------
  \newpage
  \newpage
  \begin{appendix}
    \listoffigures
    \listoftables
  \end{appendix}

\end{document}